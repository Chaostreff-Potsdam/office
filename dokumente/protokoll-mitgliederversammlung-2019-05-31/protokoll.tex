\documentclass[a4paper, 10pt, headings=normal]{scrartcl}
\KOMAoptions{twoside=true}

\usepackage[left=25mm, right=25mm, top=25mm, bottom=35mm, bindingoffset=10mm]{geometry}

\usepackage{polyglossia}
\setmainlanguage[spelling=new, latesthyphen=true]{german}

\usepackage{fontspec}
\setmainfont{Barlow}[Numbers={Lining}, BoldFont=* SemiBold]
\setsansfont{Barlow}[Numbers={Lining}, BoldFont=* SemiBold]
\newfontface\textinputfont{Barlow Condensed}

\setlength{\columnsep}{5mm}

\usepackage{multicol}

\usepackage{microtype}
\linespread{1.25}
\addtokomafont{disposition}{\linespread{1}}

\usepackage{graphicx}

\usepackage{enumitem}

\usepackage{amssymb}

\usepackage{tcolorbox}
\definecolor{light-gray}{RGB}{238, 238, 238}
\definecolor{medium-gray}{RGB}{211, 211, 211}
\definecolor{alert}{RGB}{255, 73, 43}

\usepackage{mdframed}

\usepackage{calc}

\newenvironment{textinput}[1]%
{%
	\par%
	\vspace{0.75mm}%
	\noindent%
	\linespread{1}%
	\begin{tcolorbox}[colback=light-gray, sharp corners=all, boxrule=0pt, colframe=light-gray, left=1mm, top=0.3mm, bottom=-0.3mm, right=1mm, width=#1]
		\begin{minipage}[t]{\linewidth}%
			\textinputfont\strut%
}%
{%
			\strut%
		\end{minipage}%
	\end{tcolorbox}
	\vspace{0.75mm}%
}

\newmdenv
[
	topline=false,
	bottomline=false,
	rightline=false,
	linecolor=medium-gray,
	linewidth=2pt,
	usetwoside=false,
	innerleftmargin=8.4pt,
	innerrightmargin=0pt
]{address}

\newcommand{\votesign}[1]{\includegraphics[trim=0pt 0.15pt 0pt 0pt]{vote-#1.pdf}}
\newcommand{\votesingle}[2]{\votesign{#1}~#2}
\newcommand{\vote}[3]{\votesingle{yes}{#1}, \votesingle{no}{#2}, \votesingle{abstention}{#3}}

\title{Beschlussprotokoll \\ der außerordentlichen Mitgliederversammlung \\ des \emph{Chaos Computer Club Potsdam} e.~V. (CCCP) \\ am 31.~Mai 2019}
\author{}
\date{}

\begin{document}

\maketitle
\pagestyle{myheadings}
\markboth{Protokoll der Gründungsversammlung des \emph{Chaos Computer Club Potsdam} e.~V. (CCCP)}{Protokoll der Gründungsversammlung des \emph{Chaos Computer Club Potsdam} e.~V. (CCCP)}

%\begin{multicols}{2}

\noindent Am heutigen Freitag, dem 31.~Mai 2019, versammelten sich um {19:03}~Uhr im
\begin{address}\strut%
	\textbf{freiLand Potsdam,} Haus 5

	\noindent Friedrich-Engels-Str.~22

	\noindent 14473~Potsdam
\end{address}
vier Mitglieder des Chaos Computer Club Potsdam e.~V.:

\begin{itemize}
	\item Patrick Lühne,
	\item Nico Kay Pritzkow,
	\item Sven Köhler,
	\item Matthias Jacob und
	\item Felix Grzelka (ab {19:06} Uhr).
\end{itemize}

Laut Satzung §~5 Abs.~4 war die Mitgliederversammlung beschlussfähig.

Um {19:03}~Uhr begrüßte Sven Köhler die Erschienenen und eröffnete die Mitgliederversammlung des \emph{Chaos Computer Club Potsdam} e.~V.
Herr Köhler schlug vor, die Versammlungsleitung zu übernehmen und Herrn Lühne zum Protokollführer zu bestimmen.
Dieser Vorschlag wurde einstimmig angenommen.

Herr Köhler (nachfolgend \emph{Versammlungsleiter} genannt) schlug als Wahlverfahren für alle Abstimmungen das Abstimmen durch Handzeichen vor (im Nachfolgenden stehen \votesign{yes} für Zustimmung, \votesign{no} für Ablehnung und \votesign{abstention} für Enthaltung).
Auch diesem Vorschlag stimmten allen Anwesenden zu.

Der Versammlungsleiter schlug die folgende Tagesordnung vor.

\section*{Tagesordnung}

\begin{enumerate}
	\item
		Begrüßung durch den Vorstand
	\item
		Wahl der Versammlungsleiter und Protokollführer
	\item
		Vorbehaltsbeschluss für die Satzungsänderung zur Umbenennung des Vereins
	\item
		Aussetzen der Mitgliedsbeiträge im Mai 2019
	\item
		Sonstiges
\end{enumerate}
Über die Tagesordnung wurde abgestimmt.
Die Anwesenden stimmten der Tagesordnung einstimmig zu ({\vote{4}{0}{0}}).

Um {19:06}~Uhr betrat Felix Grzelka die Mitgliederversammlung.

\section{Vorbehaltsbeschluss für die Satzungsänderung zur Umbenennung des Vereins}

Herr Köhler schlug vor, den Verein unter Vorbehalt umzubenennen.
Als neuen Namen schlug er \emph{Chaostreff Potsdam} e.~V. und die Kurzform CCCP vor.
Zu diesem Zweck schlug er vor, die Namensänderung im Titel sowie §~1 Abs.~1 der Satzung umzusetzen.

Als Beweggrund für die Umbenennung nannte Herr Köhler noch zu klärende Unstimmigkeiten mit dem Chaos Computer Club e.~V.
Aus diesem Grund schlug Herr Köhler vor, die Umbenennung des Vereins nur dann umzusetzen, falls der 1.~Vorstand Christoph Philipp Sterz in einer Diskussion mit dem Chaos Computer Club eine Notwendigkeit für die Umbenennung feststellt und diese dem Vorstand mitteilt.
%Zu diesem Zwecke wird eine Rückmeldung von Herrn Sterz bis zum 3. Juni erwartet.
Die Vereinsumbenennung soll nur in Kraft treten, falls bis Ablauf des 3.~Juni 2019 keine Rückmeldung seitens Herrn Sterz vorliegt oder Herr Sterz vor Ablauf des 3.~Juni 2019 eine explizite Notwendigkeit für die Umbenennung des Vereins festgestellt und dem Vorstand mitgeteilt hat.

Die Anwesenden stimmten einstimmig für die besprochene, vorbehaltliche Umbenennung des Vereins ({\vote{5}{0}{0}}).
Die laut Satzung §~8 Abs.~1 notwendige Dreiviertelmehrheit im Falle einer Satzungsänderung wurde erreicht.

\section{Aussetzen der Mitgliedsbeiträge im Mai 2019}

Herr Köhler schlug vor, im Mai 2019 auf die Mitgliedsbeiträge der Mitglieder zu verzichten.
Dem Vorschlag stimmten die Versammlungsmitglieder einstimmig zu ({\vote{5}{0}{0}}).

Außerdem schlug Herr Köhler vor, die Mitgliedsbeitrage für den Juni 2019 zurückzustellen, bis ein Vereinskonto erstellt wurde und die Zahlungsdaten für Überweisungen den Mitgliedern mitgeteilt wurden.
Auch diesem Vorschlag stimmten die Versammlungsmitglieder einstimmig zu ({\vote{5}{0}{0}}).

\section{Sonstiges}

Herr Köhler dankte den Mitgliedern für ihre Mitwirkung und fragte, ob es weitere Anträge oder Fragen gebe.
Nachdem dies nicht der Fall war, schloss er die Versammlung um {19:23}~Uhr.

\bigskip

\begin{textinput}{0.5\linewidth - 2.5mm}%
	Ort, Datum%
\end{textinput}%

\noindent%
\begin{minipage}[c]{0.5\linewidth - 2.5mm}%
	\begin{textinput}{\linewidth}%
		Unterschrift Protokollführer\\%
		Patrick Lühne%
		\vspace{16mm}%
	\end{textinput}%
\end{minipage}%

\end{document}
