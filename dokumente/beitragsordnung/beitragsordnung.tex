\documentclass[a4paper, 10pt, headings=normal]{scrartcl}
\KOMAoptions{twoside=true}

\usepackage[left=25mm, right=25mm, top=25mm, bottom=35mm, bindingoffset=10mm]{geometry}

\usepackage{polyglossia}
\setmainlanguage[spelling=new, latesthyphen=true]{german}

\usepackage{fontspec}
\setmainfont{Barlow}[Numbers={Lining}, BoldFont=* SemiBold]
\setsansfont{Barlow}[Numbers={Lining}, BoldFont=* SemiBold]
\newfontface\textinputfont{Barlow Condensed}

\setlength{\columnsep}{5mm}

\usepackage{multicol}

\usepackage{microtype}
\linespread{1.25}
\addtokomafont{disposition}{\linespread{1}}

\renewcommand\sectionformat{§~\thesection\enskip}

\usepackage{graphicx}

\usepackage{enumitem}

\usepackage{amssymb}

\usepackage{tcolorbox}
\definecolor{light-gray}{RGB}{238, 238, 238}
\definecolor{alert}{RGB}{255, 73, 43}

\newenvironment{textinput}[1]%
{%
	\par%
	\vspace{0.75mm}%
	\noindent%
	\linespread{1}%
	\begin{tcolorbox}[colback=light-gray, sharp corners=all, boxrule=0pt, colframe=light-gray, left=1mm, top=0.3mm, bottom=-0.3mm, right=1mm, width=#1]
		\begin{minipage}[t]{\linewidth}%
			\textinputfont\strut%
}%
{%
			\strut%
		\end{minipage}%
	\end{tcolorbox}
	\vspace{0.75mm}%
}

\title{Beitragsordnung des \\ \emph{Chaos Computer Club Potsdam} e.~V. (CCCP)}
\author{Stand: 10.~April 2019}%}
\date{}

\begin{document}

\maketitle
\markboth{Beitragsordnung des \emph{Chaos Computer Club Potsdam} e.~V. (CCCP)}{Beitragsordnung des \emph{Chaos Computer Club Potsdam} e.~V. (CCCP)}

\thispagestyle{empty}

\begin{multicols*}{2}

\section{Höhe der Beiträge (gestaffelt)}

\begin{enumerate}[label={(\arabic*)}]
	\item
		\label{itm:beitrag-regulaer}
		Der Mitgliedsbeitrag beträgt \textbf{23,42~€} pro Monat.
	\item
		\label{itm:beitrag-student}
		Der Mitgliedsbeitrag für Studierende beträgt \textbf{13,37~€} pro Monat.
		Er erfordert die Vorlage einer Studienbescheinigung.
	\item
		Der ermäßigte Mitgliedsbeitrag beträgt \textbf{9,00~€} pro Monat.
		Die Ermäßigung gilt für Schüler, Auszubildende, Arbeitslose, Rentner, Erwerbsgeminderte und Schwerbehinderte.
		Er erfordert die Vorlage eines schriftlichen Nachweises.
		Der ermäßigte Beitrag ist in weiteren Fällen nur durch begründete Anfrage zu erwerben.
		Der Vorstand entscheidet über die Annahme dieser Begründung.
	\item
		Der Mitgliedsbeitrag für Personen unter 16 Jahren beträgt \textbf{2,56~€}.
	\item
		\emph{Erhöhter Beitrag:} Es ist den Mitgliedern freigestellt, den Beitrag über den in \ref{itm:beitrag-regulaer} und~\ref{itm:beitrag-student} genannten monatlichen Beitrag zu erhöhen.
		Mehreinzahlungen werden als erhöhter Beitrag aufgefasst.
\end{enumerate}

\section{Fälligkeit}

Der Mitgliedsbeitrag wird jeweils zum 1.~des Monats sowie mit der Annahme des Aufnahmeantrags fällig.

\section{Zahlungsweise}

\begin{enumerate}[label={(\arabic*)}]
	\item
		\label{itm:ueberweisung}
		Die Zahlung des Beitrages erfolgt monatlich auf das Konto des \emph{Chaos Computer Club Potsdam} e.~V.
		Es wird empfohlen, einen Dauerauftrag einzurichten.
	\item
		Alternativ zu \ref{itm:ueberweisung} kann nach Absprache eine Barzahlung an den Schatzmeister erfolgen.
\end{enumerate}

\section{Säumnis}

Im Säumnisfall wird das Mitglied nach dreimonatigem Ausbleiben des Beitrags schriftlich gemahnt.
Zahlt ein Mitglied trotz zweifacher Mahnung oder länger als drei Monate den Beitrag nicht, so gilt nach Ablauf eines Monates nach der zweiten Mahnung die Nichtzahlung als Austritt.
In der zweiten Mahnung ist auf die Folgen der Nichtzahlung hinzuweisen.

\section{Aufnahmegebühren}

Aufnahmegebühren werden nicht erhoben.

\section{Erstattungen}

Eine Erstattung von Mitgliedsbeiträgen findet nicht statt.

\end{multicols*}

\end{document}
